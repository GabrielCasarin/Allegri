\documentclass[a4paper,12pt]{report}

\usepackage[brazil]{babel}
\usepackage[utf8]{inputenc}
\usepackage[T1]{fontenc}

\usepackage[top=2cm,bottom=2.5cm,left=2.5cm,right=2cm]{geometry}

\begin{document}

    \begin{titlepage}
        \begin{center}

            % \large
            UNIVERSIDADE DE SÃO PAULO\\
            Escola Politécnica\\
            Departamento de Computação e Sistemas Digitais
            \vspace{8cm}
            
            \Huge
            \textbf{Compilador Allegri}
            
            \vspace{0.5cm}
            \large
            Projeto Final da Disciplina de Compiladores - PCS 2508
            
            \vspace{2.5cm}
            \Large
            Gabriel Casarin da Silva
            
        \end{center}
        
        \vspace{3.0cm}
        \setlength{\parindent}{10.5cm}
        \large Professor Responsável:

        \setlength{\parindent}{10.5cm}
        Prof. Dr. João José Neto
        

        \begin{center}
            \vfill
            \large
            São Paulo, 2016
        \end{center}
            
    \end{titlepage}
        

    \tableofcontents
    % \renewcommand{\labelitemi}{\textendash}
    \addcontentsline{toc}{section}{Introito}
    \newpage
    \chapter*{Introito}
    Este é o relatório do projeto da disciplina PCS 2508 - Linguagens e Compiladores de 2016.
    O projeto consistiu em desenvolver uma linguagem de programação e projetar um compilador para ela.

    A linguagem desenvolvida foi denominada Barber. O seu compilador, Allegri.

    Por isso, este documento é divido em duas partes. A primeira descreve a linguagem Barber, sua motivação e aspéctos teóricos. A segunda documenta as etapas do desenvolvimento do compilador Allegri.

    Ao final, trazemos uma série de programas-exemplo e de testes dos códigos de máquina gerados pela compilação realizados no simulador de Máquina de von Neumann (MVN).
    
    \addcontentsline{toc}{part}{Um Prelúdio à Linguagem Barber}
    \part*{Um Prelúdio à Linguagem Barber}

    \addcontentsline{toc}{section}{Motivação}
    \chapter*{Motivação}
    
    \part*{Compilador Allegri\\\textit{Da capo}}
    \addcontentsline{toc}{part}{Compilador Allegri - Da capo}
    \chapter*{Arquitetura}
    \addcontentsline{toc}{section}{Arquitetura}
    \chapter*{Análise Léxica}
    \addcontentsline{toc}{section}{Análise Léxica}
    \chapter*{Análise Sintática}
    \addcontentsline{toc}{section}{Análise Sintática}
    \chapter*{Análise Semântica e Geração de Código}
    \addcontentsline{toc}{section}{Análise Semântica e Geração de Código}
    \subsection*{Organização da Memória}
    \addcontentsline{toc}{subsection}{Organização da Memória}

    \part*{Testes e Simulações\\\textit{Scherzo}}
    \addcontentsline{toc}{part}{Testes e Simulações - Scherzo}
    \chapter*{Finale}
    \addcontentsline{toc}{section}{Conclusões}
\end{document}
